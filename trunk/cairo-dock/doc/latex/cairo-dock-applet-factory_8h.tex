\section{Référence du fichier /opt/cairo-dock/trunk/cairo-dock/src/cairo-dock-applet-factory.h}
\label{cairo-dock-applet-factory_8h}\index{/opt/cairo-dock/trunk/cairo-dock/src/cairo-dock-applet-factory.h@{/opt/cairo-dock/trunk/cairo-dock/src/cairo-dock-applet-factory.h}}
\subsection*{Fonctions}
\begin{CompactItemize}
\item 
cairo\_\-surface\_\-t $\ast$ {\bf cairo\_\-dock\_\-create\_\-applet\_\-surface} (gchar $\ast$cIconFileName, cairo\_\-t $\ast$pSourceContext, double fMaxScale, double $\ast$fWidth, double $\ast$fHeight, gboolean bApplySizeRestriction)
\item 
{\bf Icon} $\ast$ {\bf cairo\_\-dock\_\-create\_\-icon\_\-for\_\-applet} ({\bf CairoDockContainer} $\ast$pContainer, int iWidth, int iHeight, gchar $\ast$cName, gchar $\ast$cIconFileName, {\bf CairoDockModule} $\ast$pModule)
\end{CompactItemize}


\subsection{Documentation des fonctions}
\index{cairo-dock-applet-factory.h@{cairo-dock-applet-factory.h}!cairo_dock_create_applet_surface@{cairo\_\-dock\_\-create\_\-applet\_\-surface}}
\index{cairo_dock_create_applet_surface@{cairo\_\-dock\_\-create\_\-applet\_\-surface}!cairo-dock-applet-factory.h@{cairo-dock-applet-factory.h}}
\subsubsection{\setlength{\rightskip}{0pt plus 5cm}cairo\_\-surface\_\-t$\ast$ cairo\_\-dock\_\-create\_\-applet\_\-surface (gchar $\ast$ {\em cIconFileName}, cairo\_\-t $\ast$ {\em pSourceContext}, double {\em fMaxScale}, double $\ast$ {\em fWidth}, double $\ast$ {\em fHeight}, gboolean {\em bApplySizeRestriction})}\label{cairo-dock-applet-factory_8h_eae3d51600a868ecf56f95a64df24539}


Cree une surface cairo qui pourra servir de zone de dessin pour une applet. \begin{Desc}
\item[Paramètres:]
\begin{description}
\item[{\em cIconFileName}]le nom d'un fichier image a appliquer sur la surface, ou NULL pour creer une surface vide. \item[{\em pSourceContext}]un contexte de dessin; n'est pas altere. \item[{\em fMaxScale}]le zoom max auquel sera soumis la surface. \item[{\em fWidth}]largeur de la surface obtenue. \item[{\em fHeight}]hauteur de la surface obtenue. \item[{\em bApplySizeRestriction}]restreindre la taille de la surface aux valeurs autorisees dans les docks. \end{description}
\end{Desc}
\begin{Desc}
\item[Renvoie:]la surface nouvellement generee. \end{Desc}
\index{cairo-dock-applet-factory.h@{cairo-dock-applet-factory.h}!cairo_dock_create_icon_for_applet@{cairo\_\-dock\_\-create\_\-icon\_\-for\_\-applet}}
\index{cairo_dock_create_icon_for_applet@{cairo\_\-dock\_\-create\_\-icon\_\-for\_\-applet}!cairo-dock-applet-factory.h@{cairo-dock-applet-factory.h}}
\subsubsection{\setlength{\rightskip}{0pt plus 5cm}{\bf Icon}$\ast$ cairo\_\-dock\_\-create\_\-icon\_\-for\_\-applet ({\bf CairoDockContainer} $\ast$ {\em pContainer}, int {\em iWidth}, int {\em iHeight}, gchar $\ast$ {\em cName}, gchar $\ast$ {\em cIconFileName}, {\bf CairoDockModule} $\ast$ {\em pModule})}\label{cairo-dock-applet-factory_8h_32a4b2bfcc94203e18a69c14d351db01}


Cree une icone destinee a une applet. \begin{Desc}
\item[Paramètres:]
\begin{description}
\item[{\em pContainer}]le container ou sera inseree ulterieurement cette icone. \item[{\em iWidth}]la largeur desiree de l'icone. \item[{\em iHeight}]la hauteur desiree de l'icone. \item[{\em cName}]le nom de l'icone, tel qu'il apparaitra en etiquette de l'icone. \item[{\em cIconFileName}]le nom d'un fichier image a afficher dans l'icone, ou NULL si l'on souhaitera dessiner soi-meme dans l'icone. \item[{\em pModule}]le module (necessaire poru que l'icone soit consideree comme une applet lors de son remplissage). \end{description}
\end{Desc}
\begin{Desc}
\item[Renvoie:]l'icone nouvellement cree. Elle n'est \_\-pas\_\- inseree dans le dock, c'est le gestionnaire de module qui se charge d'inserer les icones renvoyees par les modules. \end{Desc}
